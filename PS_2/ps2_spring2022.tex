\documentclass[english,12pt]{article}
\usepackage[utf8]{inputenc}
\usepackage{babel}
\usepackage[dvips]{graphicx}
\usepackage{tabularx}
\usepackage{array}
\usepackage{natbib}
\usepackage{delarray}
\usepackage{dcolumn}
\usepackage{float}
\usepackage{amsmath}
\usepackage{psfrag}
\usepackage{multirow,hhline}
\usepackage{amstext}
\usepackage{pstricks}
\usepackage{pst-plot}
\usepackage{amssymb}
\usepackage{fancyhdr}
\usepackage{cite}
\usepackage{rotating}
\usepackage{textcomp}
\usepackage{natbib}
\usepackage{setspace}
\usepackage{booktabs}
\usepackage{hyperref}
\usepackage{pdflscape}
\usepackage{multirow}
\usepackage{bbding}
\usepackage{bbm}


\newcommand\doubleRule{\midrule\midrule}
\newcommand{\indep}{\perp \!\!\! \perp}
\newcommand{\notindep}{\not\indep}

\newcommand{\pr}[2]{\frac{\partial #1}{\partial #2}}
\newcommand{\re}[1]{\smallskip\textsc{Respuesta} \begin{it}\\ #1 \end{it} \bigskip}
\textwidth=480pt \textheight=610pt
\oddsidemargin=0in\topmargin=-.2in
\def\max{mathop{\mbox{\normalfont m\'ax}}\limits}
\def\min{mathop{\mbox{\normalfont m\'{i}n}}\limits}

\usepackage{multirow,hhline}



\begin{document}



\title{\sc Advanced Microeconometrics - Spring 2022\\Problem Set 2}
\author{Instructor: Cristi\'an S\'anchez}
\date{Due date: April 27, 2022}

\maketitle


\bigskip
\bigskip


%%%%%%%%%%%%%
\noindent \textit{NOTE: You are allowed to use already-written commands that are available in statistical softwares (e.g. regress in STATA, lm in R) to perform linear estimation. However, whenever I ask you to perform nonlinear estimation, I want you to explicitly code the estimation routine, and not to use user-written commands that may be available in some softwares.}\\

\bigskip

\noindent This set of questions is inspired by Rau, Sanchez and Urzua (2019). The file \textit{data.csv} contains real data for Chilean individuals, that track their schooling and labor market decisions and outcomes from their early teens through their late twenties. 

The goal is to estimate a generalized Roy model for the decision to attend a private high school in lieu of a public high school, and labor market outcomes, i.e. wages. The model includes an unobserved factor that approximates a combination of individuals' scholastic abilities.

A description of the variables in the data is as follows:
\begin{itemize}
	\item \textit{test\_X} is students' performance on standardized test X. The unit measure is standard deviations.
	\item \textit{privateHS} is a dummy indicating having attended a private high school.
	\item \textit{wage} is the natural log of wage.
	\item \textit{male}, \textit{momschoolingX}, \textit{dadschoolingX}, \textit{broken\_homeX}, \textit{incomehhX}, \textit{north}, \textit{center} are demographic variables.
	\item \textit{share\_private} and \textit{avg\_price} are instruments for the decision of attending a private high school, and denote the local share of private high schools and the average fees local private high schools charge.
\end{itemize}

Formally, the model includes potential outcomes as follows,
    \begin{eqnarray*}
    Y_{1} & = & X\beta_1 + \theta\alpha_1 + U_{1} \\
    Y_{0} &=& X\beta_0 + \theta\alpha_0 + U_{0} ,
    \end{eqnarray*}
where $Y_1$ is the potential outcome (i.e. wages) in the counterfactual of attending a private high school, and $Y_0$ is similarly defined for the counterfactual of attending a public high school. All relevant observable demographics are included in $X$, while $\theta$ is the unobserved one-dimensional factor (i.e. ability) determining labor market outcomes. The unobserved factor is normally distributed with mean zero and standard deviation $\sigma_\theta$. The terms $U_1$ and $U_0$ are idiosyncratic error terms, that are normally distributed with mean zero and standard deviations $\sigma_1$ and $\sigma_0$, respectively.

Individuals decide whether or not to attend a private high school based on a latent variable $I$:
    \begin{eqnarray*}
    I &=&Z\gamma + \theta\alpha_I + V , 
    \end{eqnarray*}
where $Z$ include observable demographics and instruments, and $V$ is an idiosyncratic error term with zero mean and unit variance. Note that the unobservable factor is also present in this part of the model. We can thus define a binary variable $D$ indicating treatment status,
    \begin{eqnarray*}
    D &=&\mathbbm{1}\left[ I\geq0\right].
    \end{eqnarray*}

The model includes a measurement system, that helps with the identification of the distribution of $\theta$. Specifically,
    \begin{eqnarray*}
    T_k &=&W\omega_k + \theta\alpha_{T_k} + \varepsilon_k \quad\quad \mbox{ for each } k=1,2,3,4 , 
    \end{eqnarray*}
where $T_k$ is test score $k$, $W$ include demographics determining test scores, and $\varepsilon_k$ is a normally distributed error term with mean zero and standard deviation $\sigma_{\varepsilon_k}$.
    
Finally, we assume that the error terms in the model are all independent from each other conditioning on the observables and the unobserved factor, i.e. $U_1 \indep U_0 \indep V \indep \varepsilon \mid X, Z, W, \theta$, and $\theta$ is independent of all observables.

For the empirical implementation of the model, in $X$ include \textit{male}, \textit{north}, and \textit{center}. In $Z$ include all variables in $X$ plus \textit{share\_private} and \textit{avg\_price}. In $W$ include \textit{male}, \textit{momschoolingX}, \textit{dadschoolingX}, \textit{broken\_homeX}, \textit{incomehhX}, \textit{north}, and \textit{center}. The outcome variable is \textit{wage}. $D$ is \textit{privateHS}. And, the measurement system is comprised by the four test scores \textit{test\_X}.


\begin{enumerate}    
	\item Normalize $\alpha_{T_1}=1$. Discuss and show a sketch of identification of the remaining loadings in the measurement system (i.e. $\alpha_{T_2}$, $\alpha_{T_3}$, $\alpha_{T_4}$) and of the distribution of $\theta$. Why do we need to assume that one of the loadings in the measurement system is equal to one?
	\item Run an OLS regression of $Y$ ($=DY_1 + (1-D)Y_0$) on $D$ and $X$. Show your results. Are they biased? Why?
	\item Using $Z$ as instruments, estimate by 2SLS the ``effect'' of $D$ on $Y$. Show and comment your results.
	\item Now, run your OLS regression in 2. but adding $T_1$--$T_4$ as additional controls. What is the effect of $D$ on $Y$?
	\item Define the MTE, ATE, and TT parameters in this framework.
	\item Write down the likelihood function of the model, integrating out the factor $\theta$ over its distribution.
	\item Estimate the model by maximum likelihood. Show your estimated coefficients and corresponding standard errors. \textit{HINT: I suggest you use a Gauss-Hermite quadrature for numerical integration; however, you are free to use any other numerical integration routine (e.g. Monte Carlo, Quasi-Monte Carlo).}
	\item Using your estimates, simulate 10,000 observations from your model. Compare simulated and actual data averages for test scores, the decision of attending a private high school, and wages conditional on the decision of attending a private high school.
	\item Using your simulated data, plot the distribution of $\theta$.
	\item Using your simulated data, plot $Y_1 - Y_0$.
	\item Using your simulated data, compute and show the ATE and TT parameters.
	\item Using your simulated data, plot the TT parameter as function of the unobserved ability, $\theta$.
\end{enumerate}





%%%%%%%%%%%%%%%%%%%%%%%%%

\end{document} 
